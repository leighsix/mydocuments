% $ based on Id: sample_english-v1.2.tex,v 1.2 2007/04/12 21:05:22 zlb Exp $
% $Id: sample_english.tex 6 2011-01-24 13:13:33Z hsqi $

\documentclass[english]{cccconf}
%\documentclass[usemulticol,english]{cccconf}
\usepackage[comma,numbers,square,sort&compress]{natbib}
\usepackage{epstopdf}
\usepackage{comment}
\usepackage{subcaption}
\usepackage{color}
\usepackage{lipsum}
\usepackage{multicol}
\usepackage{amssymb}
\usepackage{amsmath}
\usepackage{bm}
\usepackage{amsfonts}
\usepackage{multirow}
\usepackage{graphicx}
\usepackage{mathtools, amssymb}
\usepackage[comma,numbers,square,sort&compress]{natbib}
\usepackage{CJK}
\usepackage{graphicx}
\usepackage{epsfig}
\usepackage{float}
\usepackage{multirow}
\usepackage{algorithm}
\usepackage{algorithmic}
\usepackage{indentfirst}
\usepackage{booktabs}
\usepackage{amsmath}

\begin{document}

\title{Competition of Social Opinions on Two Layer Networks}

% Note: the first argument in the \affiliation command is optional.
% It defines a label for the affiliation which can be used in the \aref
% command. If there is only one affiliation for all authors, then the
% optional argument in the \affiliation command should be suppressed,
% and the \aref command should also be removed after each author in
% \author command, in this case the affiliation will not be numbered.
% \author{First Author, Second Author, Third Author}
% \affiliation{Chinese Academy of Sciences, Beijing 100190, P.~R.~China\email{ccc@amss.ac.cn}}

\author{Cho Hyunchel \aref{amss,hit},
        A. Mahmood \aref{amss,hit},
        Lin Wang \aref{amss,hit}}
\affiliation[amss]{Department of Automation, Shanghai Jiao Tong University, Shanghai 200240, P.~R.~China}
\affiliation[hit]{Key Laboratory of System Control and Information Processing, Ministry of Education of China, Shanghai 200240, P.~R.~China
        \email{leighsix@naver.com}\email{arfan499@sjtu.edu.cn}\email{wanglin@sjtu.edu.cn}}

\maketitle

\begin{abstract}
Social conflict usually can be investigated based on competition on two-layers network. In this paper, a competition model is studied on interconnected networks with two-layer opinions, where the first layer is opinion formation and the second layer is decision making. Starting with a polarized competition case where layer A has all positive opinion and layer B has all negative opinion, competition simulations are considered based on different network structures. With Monte Carlos simulations, different structural models are estimated and compared with average state and consensus ratio of two layers. That shows internal and external degrees play the vital role for performing consensus. Especially, increasing the number of external degrees and internal degrees on one side layer make it easy to have consensus state. However, too many edges on two-layers make it hard to make consensus due to inner conflict.
\end{abstract}

\keywords{Interconnected Networks, Opinion Dynamics, Decision Making}

% Please remove or comment out the following line if the footnote is not necessary
\footnotetext{This work was supported by the National Natural Science Foundation of China under Grant Nos 61873167, 61473189, the Natural Science Foundation of Shanghai (No.17ZR1445200).}

\section{Introduction}
In various situations ranging from voting to adoption of new policies, it is widely recognized that opinion formation and decision making formation have mutual interaction as interconnected networks\cite{bianconi2018,domenico2013,tomasini2015, kimsangwoo2012,newman2010,boccaletti2014,mikko2013,huberman2004}. Many researchers have devoted to modeling and analyzing competition on opinion dynamics\cite{amato2017,quattrociocchi2014,haibo2017, hua2014}, voter model\cite{redner2017}, game theory\cite{smyrnakis2019} and many more\cite{danziger2019,namkhanhvu2017,laguna2004,masuda2015,zuev2012, shenyu2018}.  
 
For competition of interconnected networks, many researches have been performed in the various networks, for example the dissemination of computer viruses, messages, opinions, memes, diseases and rumors\cite{hua2014,shenyu2018,alvarez2016,gomez2015,diep2017,rocca2014,velasquez2018}. Opinion dynamics on two-layer or multi-layer networks are investigated, based on \textit{Abrams-Strogatz(AS)} model\cite{abrams2003,vazquez2010} and $M$ model\cite{rocca2014}. Existing researches mainly focused on that under what conditions all agents reach a consensus or dissent, which have shown that the system can make positive consensus, negative consensus or coexistence under certain range of volatility, reinforcement, or prestige. Also, the threshold or critical point for transition are found out to explain and analyze the social phenomena in real world such as the legislation, election result, and social network\cite{alvarez2016, amato2017, diep2017}. In \cite{gomez2015}, it is shown that the transition from localized to mixed status occurs via a cascade from poorly connected nodes in layers to those highly connected ones. In addition, the main features, such as transition and cascade, found in Monte Carlo simulation is exactly characterized by the mean-field theory and magnetization\cite{alvarez2016, diep2017, amato2017, gomez2015}.   

In this paper, we investigate the competitions on two interconnected networks with various different structures, considering which structure has more probability to perform consensus results. With Monte Carlos simulations, consensus of two layers would be estimated and compared with different structural models, which show the vital influence of internal and external degrees. We provide three results from these simulations. First, when the number of external degrees in decision making layer is more than the other layer, the tendency to make consensus on two-layers is stronger. Second, the more the average number of internal degrees in one layer is, the stronger the tendency to keep and maintain the state of the layer is. Third, when each layer has lots of internal degrees individually, it is hard to make consensus due to inner conflict.    

The paper is organized as follows. In section 2, competition dynamics of interconnected network, that is applied to each layer, are described. In section 3, the simulation results of different structural networks are presented. Finally, in section 4, the simulation results will be summarized.

\section{Modeling}
The model consists of two layers, and each layer has different dynamics. For layer A, the node change its states according to $M$ model as introduced in \cite{rocca2014}. Here, we choose $M=2$, that each node has four states $(-2, -1, +1, +2)$. For each link $(k, j)$ belong to layer A,  the dynamics are designed as follows:
\begin{itemize}
\item Compromise : if they have opposite orientations, their states become more moderate with probability $q$ :
\begin{align*}
\mbox{if } S_k<0 \mbox{ and } S_j>0  \Rightarrow (S_k, S_j) \rightarrow (S_k^r, S_j^l) \mbox{ with } prob.q,\\
\mbox{if } S_k>0 \mbox{ and } S_j<0  \Rightarrow (S_k, S_j) \rightarrow (S_k^l, S_j^r) \mbox{ with } prob.q.
\end{align*}
If $S_k = \pm1$ and $S_j = \mp1$, one switches orientation at random:
\begin{align*}
(\pm 1, \mp 1)\rightarrow \left\{\begin{matrix}
(+1, +1) \mbox{ with } prob.q/2,
\\(-1, -1)\mbox{ with } prob.q/2.
\end{matrix}\right.
\end{align*}
\item Persuasion : if they have the same orientation, their states become more extreme with probability $p$ :
\begin{align*}
\mbox{if } S_k<0 \mbox{ and } S_j<0  \Rightarrow (S_k, S_j) \rightarrow (S_k^l, S_j^l) \mbox{ with } prob.p,\\
\mbox{if } S_k>0 \mbox{ and } S_j>0  \Rightarrow (S_k, S_j) \rightarrow (S_k^r, S_j^r) \mbox{ with } prob.p.
\end{align*}
\end{itemize}
For each external link $(k,j)$ with $k$ belong to layer A, the state of node $k$ is updated according to :
\begin{itemize}
\item $S_k \times S_j < 0$ :
\begin{align*}
\mbox{if } S_k<0 \mbox{ and } S_j>0  \Rightarrow (S_k, S_j) \rightarrow (S_k^r, S_j) \mbox{ with } prob.q,\\
\mbox{if } S_k>0 \mbox{ and } S_j<0  \Rightarrow (S_k, S_j) \rightarrow (S_k^l, S_j) \mbox{ with } prob.q.
\end{align*}
\item $S_k \times S_j > 0$ :
\begin{align*}
\mbox{if } S_k<0 \mbox{ and } S_j<0  \Rightarrow (S_k, S_j) \rightarrow (S_k^l, S_j) \mbox{ with } prob.p,\\
\mbox{if } S_k>0 \mbox{ and } S_j>0  \Rightarrow (S_k, S_j) \rightarrow (S_k^r, S_j) \mbox{ with } prob.p.
\end{align*}
\end{itemize}
Here, $S_k^r$ and $S_k^l$ denote the right and left neighboring states of $k$, defined as
\begin{align*}
S_k^r &= \left\{\begin{matrix}
1,\mbox{ for } S_k = -1\\
2,\mbox{ for } S_k = +2\\ 
S_k + 1,\mbox{ otherwise }, 
\end{matrix}\right. &
S_k^l &= \left\{\begin{matrix}
-1,\mbox{ for } S_k= +1
\\ -2,\mbox{ for } S_k=-2
\\ S_k - 1,\mbox{ otherwise }.
\end{matrix}\right.
\end{align*}


The sign of $S^A$ represents its opinion orientation and its absolute value $|S^A|$ measures the intensity of its opinion. So, $|S^A|=2$ represents to a positive or a negative extremist, while  $|S^A|=1$ correspond to a moderate opinion of each side. In case of internal link $(k, j)$ belong to layer A, if the nodes are the same orientation$(S_kS_j>0)$, then with probability $p$ the states of nodes become extreme$(S_k=\pm1 \rightarrow \pm2, S_j= \pm1 \rightarrow \pm2)$ if they are moderate, or remain extreme if they are already extreme$(S_k=\pm2 \rightarrow \pm2, S_j= \pm2 \rightarrow \pm2)$. Inversely if the nodes are opposite orientations$(S_kS_j<0)$, with probability $q$ the states of nodes become moderate it they are extreme$(S_k=\pm2 \rightarrow \pm1, S_j= \pm2 \rightarrow \pm1)$, or switch orientations if they are already moderate$(S_k=\pm1 \rightarrow \mp1, S_j= \pm1 \rightarrow \mp1)$.  In case of interaction between layer A node and layer B node, layer A node follows opinion dynamics formula, but the state of layer B node does not change. In other words, the state of layer B affects layer A, but this dynamics does not affect the state of layer B node. For example, one of layer A node, $S_k = +2$ is connected with one of layer B node, $S_j = -1$. In this case, $S_k$ will change into $S_k = +1$ with $prob.q$. But $S_j$ will not change, which indicates that the states of layer B will influence the states of layer A.

The dynamics of layer B follows the decision-making dynamics as introduced in \cite{abrams2003, vazquez2010}. The state of node i in layer B can be +1 and -1, and it updates according to

\begin{equation}
P_B(S_i \rightarrow -S_i)= \left \{\frac{n^{-S_i}}{i_i + e_i}\right \}^\beta,
\end{equation}

where $i_i$ is the number of internal edges and $e_i$ is the number of external edges. $n^{-S_i}$ is the number of neighbors of i with opposite state $-S_i$. $\beta(\geq 0)$ is the volatility exponent that measures how prone a node change its state. If $\beta \simeq 0$, a node is very likely to change its state. Inversely, if $\beta \gg 1$, a node is unlikely to change its state. Also, this formula means that the more the number of nodes connected with the opposite state is, the easier the nodes are to change into the opposite state.\\
\begin{figure}[!htb]
  \centering
  \includegraphics[width=\hsize]{FIG1.png}
  \caption{Competition of Interconnected Network}
  \label{Fig1}
\end{figure}


\section{Simulations and Analysis}
To start with a polarized competition, as the initial conditions,  nodes in layer A are all positive, and nodes in layer B are all negative. As nodes in layer A can be various states, it begins with the status where an half of nodes are $+1$ and the others are $+2$. And the nodes of layer B have only $-1$. 

There are two parameters in the dynamics of layer A. To simply represent the probability $p$ and probability $q$ together, we set $\gamma = p/q$ based on $p+q=1$. Here, $\gamma$ represents the tendency of opinion such as extreme or moderate, which is scaled to be 0 to 2. However, the scale of $\beta$, in the dynamics of layer B, depends on the total number of degrees. 

To implement the interconnected dynamics, one step consists of two layers dynamics, where for every link with at least one node in layer A will be checked, and every node in layer B will updates its state according to the decision-making dynamics. Each simulation takes 100 steps, and 100 simulations are considered  for each set of parameters. In the following simulations, we use \textit{`Average State'(AS)} to measure the competition result.

\begin{equation}
AS = avg\left( {\sum\limits_i^{{K^A}} {S_i^A/4} } \right) + avg\left( {\sum\limits_i^{{K^B}} {S_i^B/2} } \right).
\end{equation}

With \textit{AS}, it could be checked whether the consensus happens or not in accordance with $\gamma$ and $\beta$ changing.  If the positive consensus happens, it would be close to the value of $+1$ and if the negative consensus happens, it would be close to the value of $-1$. The values between $+1$ and $-1$ are not on the consensus yet, so these states are considered as belonging to the coexistence part.

\begin{figure*}[!htb]
	\centering
	\includegraphics[width=\hsize]{FIG2.png}
	\caption{(a) $\gamma$-\textit{AS} chart according to certain $\beta$ values. (b) $\beta$-\textit{AS} chart according to certain $\gamma$ values.}
	\label{Fig2}
\end{figure*}

To estimate and evaluate the consensus results regarding all $\gamma$s and all $\beta$s, we use four kinds of measures including \textit{`AS total'}, \textit{`Positive Consensus Ratio'(PCR)}, \textit{`Negative Consensus ratio'(NCR)}, and \textit{`Consensus Ratio'(CR)}. \textit{AS total} means the summation of \textit{AS} for all $\gamma$s and all $\beta$s. In formula(3), ${A{S_{{\gamma _i},{\beta _j}}}}$ means $AS$ value with parameters, $\gamma_i$ and $\beta_j$. It could show the total orientation and intensity when considering different networks. \textit{PCR} is the ratio of positive consensus in the simulation result. When ${A{S_{{\gamma _i},{\beta _j}}} \simeq  1}$, it is considered as positive consensus, the number of simulation results with positive consensus is counted, and the ratio to the number of all experiments under same network structure is calculated as \textit{PCR}. Similarly, \textit{NCR} is the ratio of experiments with negative consensus. \textit{CR} is the ratio of experiments reaching consensus, i.e. summation of \textit{PCR} and \textit{NCR}.

\begin{equation}
\begin{array}{cl}
AS\mbox{ \textit{total} } = \frac{{\sum\limits_j^m {\sum\limits_i^n {A{S_{{\gamma _i},{\beta _j}}}} } }}{{n \times m }}, &
\begin{array}{l}
\gamma  = \left\{ {{\gamma _{\rm{1}}},{\gamma _{\rm{2}}},\left. {\cdot\cdot\cdot,{\gamma _n}} \right\}} \right.\\
\beta {\rm{ = }}\left\{ {{\beta _{\rm{1}}},{\beta _{\rm{2}}},\left. {\cdot\cdot\cdot,{\beta _m}} \right\}} \right.
\end{array}.\
\end{array}
\end{equation}

\begin{equation}
PCR = \frac{{\sum\limits_j^m {\sum\limits_i^n {(A{S_{{\gamma _i},{\beta _j}}} \simeq  1)} } }}{{n \times m}}.
\end{equation}

\begin{equation}
NCR = \frac{{\sum\limits_j^m {\sum\limits_i^n {(A{S_{{\gamma _i},{\beta _j}}} \simeq   - 1)} } }}{{n \times m}}.
\end{equation}

\subsection{Competition on Random Regular Networks}
In this subsection, each layer consists of random regular network that has $N$ nodes with $k$ internal edges as introduced in \cite{kimsangwoo2012, bela2001}. Each node of one layer is connected with a random node on the other layer. That means each node has only $1$ external un-directed edge. Basically, the simulations are done with $N=2048$, and $k=5$. 

The simulation results are shown in Fig.~\ref{Fig2} and Fig.~\ref{Fig3}. Fig.~\ref{Fig2}(a) shows that when $\gamma$ increases, if $\beta$ is in some range$(1.2 < \beta < 1.95)$, it normally tends to positive consensus. But, if $\beta$ is lower or larger than some values, it doesn't make consensus.
In Fig.~\ref{Fig2}(b), as $\beta$ increases, it normally change from positive to negative consensus. But, when $\gamma$ is very low($\gamma \le 0.1$), it doesn't make positive consensus. On the other hand, when $\gamma$ is large enough, it makes positive consensus. But, when $\beta$ is large enough, it is changed into negative consensus. When both of $\gamma$ and $\beta$ are large enough, the state is in a coexistence part.
 
\begin{figure}[!htb]
	\centering
	\includegraphics[width=\hsize]{FIG3.png}
	\caption{Random Regular Networks : \textit{AS} changing with $\gamma$ and $\beta$}
	\label{Fig3}
\end{figure}

Fig.~\ref{Fig3} shows the states of two layers according to all $\gamma$s and all $\beta$s. The $X$-axis is the $\gamma$ and the $Y$-axis is the $\beta$, and the $Z$-axis represents \textit{AS}. The closer the color is to blue, the more it has positive consensus. And the closer the color is to red, the more it has negative consensus. A light and white areas have coexistence with positive states and negative states mixed. This chart has two areas for coexistence, when $\beta$ is very low or very high. When $\beta$ is in some range, interconnected network can perform positive or negative consensus with different $\gamma$ values.     

\subsection{Competition on Networks with different number of external links}

In this subsection, we consider the influence of external links. Based on the basic model in Subsection 3.1, we reduce the number of nodes in layer B at a certain rate and increase the external links from nodes in layer B accordingly.  We denote \textit{HM(n)} as a hierarchical model with a level $n$, which means that the number of nodes in layer B is $1/n$ of the number of nodes in layer A, and the number of external links from node in layer B is $n$ in view that the number of external links from node in layer A is $1$. In other words, each node in layer A has one external edge, but each node in layer B has $n$ external edges for \textit{HM(n)}, which means one node in layer B can be influenced by $n$ nodes in layer A. $\gamma$ scale is same as the Random Regular Networks Model. But, $\beta$ scale depends on the number of degrees. So the $\beta$ scale is adjusted to have the same probability of volatility with Random Regular Networks Model(\textit{RRM}) as following formula,
\begin{equation}
{\beta _{h,\max}} = {\beta _{rr,\max}} \cdot \log \left( {\frac{{{n_{rr}}^{ - {S_i}}}}{{{i_{rr,i}} + {e_{rr,i}}}} \cdot \frac{{{i_{h,i}} + {e_{h,i}}}}{{{n_{h}}^{ - {S_i}}}}} \right). 
\end{equation}

This formula is derived from formula(1). $\beta _{h,\max}$ is the maximum value of $\beta$ scale in \textit{HM}, and $\beta _{rr,\max}$ is the maximum value of $\beta$ scale in \textit{RRM}. When \textit{RRM} begins with initial state and the maximum of $\beta$ scale, it has the lowest volatility except $0$. For starting simulation under same probability in layer B dynamics, maximum value of $\beta$ in \textit{HM} is calculated when \textit{RRM} has the lowest volatility from initial state. 

Fig.~\ref{Fig4} shows the Hierarchical Model simulation results. Comparing \textit{HMs} with \textit{RRM}, \textit{CR} and \textit{PCR} are all increased remarkably. \textit{HMs} have more positive consensus part than \textit{RRM}). It shows that as the number of B nodes are decreased, it is easy to make positive consensus. Comparing \textit{HM(16)} with other \textit{HMs}, \textit{HM(16)} has the most positive consensus part. In case of models where the number of nodes in layer B is less than \textit{HM(16)},  \textit{CR} and \textit{PCR} of the models are decreased and \textit{NCR} is increased slightly. Also, in case of models where the number of nodes in layer B is more than \textit{HM(16)}, \textit{CR} and \textit{PCR} are also decreased. However, \textit{HM(4)} has the most \textit{AS total}. Although \textit{HM(4)} doesn't have the most consensus part, it has more intensity for positive social opinion. It can be analyzed that strong social intensity always do not make more consensus. These results indicate network structure can contribute to make more consensus result. 
   
\begin{figure}[!htb]
	\centering
	\includegraphics[width=\hsize]{FIG4.png}
	\caption{Hierarchical Model(\textit{HM(n)})}
	\label{Fig4}
\end{figure}

In summary, all the Hierarchical Model has more consensus ratio than Random Regular Networks Model. Among \textit{HMs}, \textit{HM(16)} has the most positive consensus part. When the number of nodes in layer B is more or less than \textit{HM(16)}, \textit{CR} and \textit{PCR} are decreased. That shows there exists the effective and efficient number of nodes in the decision making layer for performing positive consensus.  

\subsection{Competition on Networks with different number of internal links}
\begin{figure}[!htb]
	\centering
	\includegraphics[width=\hsize]{FIG5.png}
	\caption{Comparison of Networks with different internal degrees(\textit{RR(n)-RR(m)}: layer A has random regular network with $n$ internal edges, layer B has random regular network with $m$ internal edges)}
	\label{Fig5}
\end{figure}
Next, the interconnected networks are simulated with different internal degrees in order to define and evaluate the influence of internal degrees. The number of internal degrees on each node is switched to $2$ or $5$.

Fig.~\ref{Fig5} shows the simulation results with changing the number of internal edges. \textit{RR(5)-RR(2)} has the most \textit{PCR}. \textit{RR(2)-RR(5)} has the most \textit{NCR}. When the number of internal edges in layer A are more than layer B, it has more positive consensus. Inversely, when the number of internal edges in layer B are more than layer A, it has relatively more negative consensus. These results provide that the number of edges on layer A has the tendency to keep positive state, and the number of edges on layer B has the tendency to keep negative state. The number of internal edges have the influence on consensus result and a layer with more internal edges has the tendency to maintain its own state. In case of networks with same internal edges, \textit{RR(2)-RR(2)} has more \textit{PCR} and \textit{AS total} than \textit{RR(5)-RR(5)}. It can be analyzed that \textit{RR(5)-RR(5)} is hard to make consensus, because it has more internal edges to cause inner conflict. Also, \textit{RR(2)-RR(2)} has less \textit{NCR} than \textit{RR(5)-RR(5)}. It shows that the number of internal edges in layer B is more sensitive than layer A. As formula(1) shows, layer B dynamics can have more various and extreme probabilities when it has more degrees. For example, in case of \textit{RR(2)-RR(2)} with $\beta = 1$, the dynamics starts with $P_B=1/3$ and in case of \textit{RR(5)-RR(5)} with $\beta = 1$, the dynamics starts with $P_B=1/6$.    

\subsection{Competition on Networks with different structures}
So far, each layer of the interconnected network consisted of random regular networks that has the same number of edges for each node. Now, the simulation would be implemented on different network structures. 

\begin{figure}[!htb]
	\centering
	\includegraphics[width=\hsize]{FIG6.png}
	\caption{Comparison of Networks with different structures}
	\label{Fig6}
\end{figure}

Here, we use \textit{Barabasi-Albert network(BA)} structure as introduced in \cite{barabasi1999}. To evaluate the influence of network structure, 5 simulations are implemented with changing network structures. The \textit{BA} network is applied for both layers or switched on each layer. And, because layer A with \textit{BA} network structure has total $10,215$ internal edges, \textit{RR(10)-RR(5)}, under the similar conditions such as the number of nodes and edges, is also simulated. The simulation results are shown in Fig.~\ref{Fig6}. The result of \textit{BA-RR} and \textit{RR(10)-RR(5)} have almost the same features. The gap of \textit{CR} is almost same(less than 0.01). The structure of network make no obvious difference of consensus results. In case of \textit{BA-BA}, the \textit{CR} has the least ratio for consensus. \textit{BA-BA} structure has lots of internal edges on each layer. Therefore, it is hard to make consensus due to inner conflict on each layer. 

\section{Conclusion}
In this work, we have considered competition on interconnected networks, where layer A is a layer of social opinion and layer B is a network representing decision making. When these two layers are connected and interacted, three final states, negative consensus, positive consensus, and coexistence appear according to $\gamma$ and $\beta$. 

Competition results are measured with \textit{AS total}, \textit{PCR}, \textit{NCR}, and \textit{CR}, which show that the numbers of internal and external edges play very important roles on consensus of interconnected networks. Especially, we provide three conclusions about the roles of edges. First, as hierarchical models show, when the number of external edges in decision making is more than opinion layer, it is easy to make consensus on two-layers.  Also, it is found out that there exists the efficient number of nodes in decision making layer for performing consensus. Second, a layer with more internal edges has more tendency to keep its own states. Third, too many internal edges on each layer can cause inner conflict, and that makes it hard to have consensus state.  

More research will be needed to make generalized model and to be applied to real social networks. We think this research can contribute to providing the analysis tool of competing social networks such as the legalization or social decision-making system. Also, it could help to solve social conflict problems by making consensus of two layer. As future work, it would be very interesting to make the generalized model for competing interconnected network and find the key nodes and edges of interconnected network.

\begin{thebibliography}{0}
\bibitem{bianconi2018}
G. Bianconi, \textit{Multilayer Networks: Structure and Function}, Oxford University Press, 2018.

\bibitem{domenico2013}
M. De Domenico et al, \textit{Mathematical Formulation of Multilayer Networks}, Physical Review X 3, 041022, 2013.

\bibitem{tomasini2015}
Marcello Tomasini, \textit{An Introduction to Multilayer Networks}, 10.13140/RG.2.2.16830.18243, 2015.

\bibitem{kimsangwoo2012}
Kim Sangwoo, \textit{Structure and dynamics of complex networks}, Yonsei Univ, 2012.

\bibitem{newman2010}
M. E. J. Newman, \textit{Networks: An Introduction}, Oxford University Press, 2010.
	
\bibitem{boccaletti2014}
S. Boccaletti et al, \textit{The structure and dynamics of multilayer networks}, Physics Reports 544, 2014.	

\bibitem{mikko2013}
Mikko Kivela et al, \textit{Multilayer Networks}, J. Complex Networks Volume 2. DOI : 10.1093/comnet/cnu016, 2013.	

\bibitem{huberman2004}
Wu F. and Huberman B.A, \textit{Social structure and opinion formation}, arXiv.org:cond-mat/0407252v3, 2004	

\bibitem{hua2014}
Hua Jun, Lin Wang, Xiaofan Wang, \textit{An information diffusion model based on individual characteristics}, 33rd Chinese Control Conference (CCC), 2014.

\bibitem{amato2017}
R Amato et al, \textit{Opinion competition dynamics on multiplex networks}, New Journal of Physics volume 19, 2017

\bibitem{quattrociocchi2014}
Walter Quattrociocchi et al, \textit{Opinion dynamics on interacting networks : media competition and social influence}, Scientific Reports:Complex networks Statistical Physics, 2014

\bibitem{haibo2017}
Haibo Hu, \textit{Competing opinion diffusion on social networks}, Royal Society Open Science volume 4, 2017

\bibitem{redner2017}
Sidney Redner, \textit{Dynamics of Voter Models on Simple and Complex Networks}, Physics and Society(physics.soc-ph), 2017

\bibitem{smyrnakis2019}
Michalis Smyrnakis et al, \textit{An evolutionary game perspective on quantised consensus in opinion dynamics}, PLoS ONE 14(1):e0209212, 2019

\bibitem{shenyu2018}
Shenyu Zhou,  Shuyang Shi, Lin Wang, \textit{Immunizations of Interacting Diseases}, 37rd Chinese Control Conference (CCC), 2018.

\bibitem{danziger2019}
Danziger, Michael M. et al, \textit{Dynamic interdependence and competition in multilayer networks}, Nature Physics Volume 15. pp.178-185, 2019.

\bibitem{namkhanhvu2017}
Nam Khanh Vu, \textit{Robustness of Interconnected Complex Networks with Directed Dependency}, Yonsei University Department of Computer Science, 2017.

\bibitem{laguna2004}
M. F. Laguna et al, \textit{The dynamics of opinion in hierarchical organizations}, arXiv.org:nlin/0404024, 2004 

\bibitem{masuda2015}
Naoki Masuda, \textit{Opinion control in complex networks}, New Journal of Physics, Volume 17, 2015.

\bibitem{zuev2012}
Zuev A S and Fedyanin, 
\textit{Models of opinion control for agents in social networks}, Automation and Remote Control Volume 73 Issue 10, pp.1753–1764, 2012 

\bibitem{alvarez2016}
Alvarez Zuzek et al, \textit{Interacting Social Processes on Interconnected Network}, PLoS ONE. 11. 10.1371/journal.pone.0163593, 2016.

\bibitem{gomez2015}
Gomez Gardenes J. et al, \textit{Layer-layer competition in multiplex complex network}, Phil. Trans. R. Soc. A 373:20150117, 2015

\bibitem{diep2017}
H.T. Diep et al, \textit{Dynamics of two-group conflicts: A statistical physics model}, Physica A: Statistical Mechanics and its Applications Volume 467. pp.183 - 199, 2017.

\bibitem{rocca2014}
C. E. La Rocca et al, \textit{The influence of persuasion in opinion formation and polarization}, Europhys. Lett. 106, 40004, pp.1-2, 2014.

\bibitem{velasquez2018}
F. Vel{\'a}zquez et al,\textit{Opinion dynamics in two dimensions: domain coarsening leads to stable bi-polarization and anomalous scaling exponents}, Physics and Society(physics.soc-ph), 2018. 

\bibitem{abrams2003}
M Abrams et al, \textit{Modelling the dynamics of language death}, Nature.424.900. DOI : 10.1038/424900a, 2003.

\bibitem{vazquez2010}
 F. V{\'a}zquez et al, \textit{Agent based models of language competition: macroscopic descriptions and order–disorder transitions}, Journal of Statistical Mechanics: Theory and Experiment P04007, 2010.

\bibitem{bela2001}
 Bela Bollobas, \textit{Random Graphs}, 2nd edition, Cambridge University Press, section 2.4: Random Regular Graphs, 2001.

\bibitem{barabasi1999}
Barabasi A. L., Albert R, \textit{Emergence of Scaling in Random Networks}, Science 286, 509, DOI: 10.1126/science.286.5439.509, 1999.

\end{thebibliography}

\end{document}



































